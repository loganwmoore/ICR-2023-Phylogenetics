%% 
%% Copyright 2007-2020 Elsevier Ltd
%% 
%% This file is part of the 'Elsarticle Bundle'.
%% ---------------------------------------------
%% 
%% It may be distributed under the conditions of the LaTeX Project Public
%% License, either version 1.2 of this license or (at your option) any
%% later version.  The latest version of this license is in
%%    http://www.latex-project.org/lppl.txt
%% and version 1.2 or later is part of all distributions of LaTeX
%% version 1999/12/01 or later.
%% 
%% The list of all files belonging to the 'Elsarticle Bundle' is
%% given in the file `manifest.txt'.
%% 
%% Template article for Elsevier's document class `elsarticle'
%% with harvard style bibliographic references

%\documentclass[preprint,12pt,authoryear]{elsarticle}

%% Use the option review to obtain double line spacing
%% \documentclass[authoryear,preprint,review,12pt]{elsarticle}

%% Use the options 1p,twocolumn; 3p; 3p,twocolumn; 5p; or 5p,twocolumn
%% for a journal layout:
%% \documentclass[final,1p,times,authoryear]{elsarticle}
%% \documentclass[final,1p,times,twocolumn,authoryear]{elsarticle}
%% \documentclass[final,3p,times,authoryear]{elsarticle}
%% \documentclass[final,3p,times,twocolumn,authoryear]{elsarticle}
%% \documentclass[final,5p,times,authoryear]{elsarticle}
 \documentclass[final,5p,times,twocolumn,authoryear]{elsarticle}

%% For including figures, graphicx.sty has been loaded in
%% elsarticle.cls. If you prefer to use the old commands
%% please give \usepackage{epsfig}

%% The amssymb package provides various useful mathematical symbols
\usepackage{amssymb}
\usepackage{lipsum}
%% The amsthm package provides extended theorem environments
%% \usepackage{amsthm}

%% The lineno packages adds line numbers. Start line numbering with
%% \begin{linenumbers}, end it with \end{linenumbers}. Or switch it on
%% for the whole article with \linenumbers.
%% \usepackage{lineno}

%% You might want to define your own abbreviated commands for common used terms, e.g.:
\newcommand{\kms}{km\,s$^{-1}$}
\newcommand{\msun}{$M_\odot}

\journal{Using Phylogeny Summary Statistics to Find Direction of Disease Transmission}


\begin{document}

\begin{frontmatter}

%% Title, authors and addresses

%% use the tnoteref command within \title for footnotes;
%% use the tnotetext command for theassociated footnote;
%% use the fnref command within \author or \affiliation for footnotes;
%% use the fntext command for theassociated footnote;
%% use the corref command within \author for corresponding author footnotes;
%% use the cortext command for theassociated footnote;
%% use the ead command for the email address,
%% and the form \ead[url] for the home page:
%% \title{Title\tnoteref{label1}}
%% \tnotetext[label1]{}
%% \author{Name\corref{cor1}\fnref{label2}}
%% \ead{email address}
%% \ead[url]{home page}
%% \fntext[label2]{}
%% \cortext[cor1]{}
%% \affiliation{organization={},
%%            addressline={}, 
%%            city={},
%%            postcode={}, 
%%            state={},
%%            country={}}
%% \fntext[label3]{}

\title{Using Phylogeny Summary Statistics to Find Direction of Disease Transmission}

%% use optional labels to link authors explicitly to addresses:
%% \author[label1,label2]{}
%% \affiliation[label1]{organization={},
%%             addressline={},
%%             city={},
%%             postcode={},
%%             state={},
%%             country={}}
%%
%% \affiliation[label2]{organization={},
%%             addressline={},
%%             city={},
%%             postcode={},
%%             state={},
%%             country={}}

\author[first]{Logan Moore}
\affiliation[first]{organization={Institute for Computing in Research},%Department and Organization
            addressline={}, 
            city={Santa Fe},
            postcode={}, 
            state={New Mexico},
            country={United States of America}}

\begin{abstract}
%% Text of abstract
The goal of this paper is to provide a thourough comparison of six different phylogenetic tree summary statistics which can be used to determine who-infected-whom when given samples from two hosts. This was done through the use of running a large dataset of simulated trees through a program which found each summary statistic for every tree. These summary statistics were then plotted and analysed.
\end{abstract}

%%Graphical abstract
%\begin{graphicalabstract}
%\includegraphics{grabs}
%\end{graphicalabstract}

%%Research highlights
%\begin{highlights}
%\item Research highlight 1
%\item Research highlight 2
%\end{highlights}


\end{frontmatter}

%\tableofcontents

%% \linenumbers

%% main text

\section{Introduction}
\label{introduction}

Phylogenetic trees can be used as a method of conveying important information regarding species and their common ancestors. They can also be an important tool in the world of virology as they can be analysed to find important information regarding disease transmission. One piece of information that can be found is the direction of disease transmission. This is found using a combination of summarizing statistics which analyse tree size, shape, and topology. The goal of this paper is to establish a comparison of each summarizing statistic and to acess if certain statistics more accuratly measure direction than others.

\section{Background}
\label{background}

Analysis of phylogenetic trees is an important area of investigation. Effective analysis could allow for key insight into the patterns and changes that take place in species and viruses. Summarizing statistics for the direction of disease transmission have also been explored. The goal is to determine who-infected-whom when given a tree with viral sequences from two hosts(one having infected the other). This can be done through the use of summary statistics which summarize certain properties of the tree.

The summary statistics that will be analysed in this paper come from an article published by Ethan O Romero-Severson and others(\cite{10.1534/genetics.117.300284}). In this paper, they described 9 summary statistics the first of which was the root label of the tree(\cite{doi:10.1073/pnas.1522930113}). This is found by examining each node and its children. To find the root label, two sisters are acessed as being either a lineage for host A or a lineage for host B. If both are in host A, then the parent node can also be labeled as A because the parent created two lineages both within A meaning that the parent themself must also be in A. The same rule applies if both sisters belong to B. If one is A and the other is B then the parent is labeled as unknown or ?. If one sister is A or B and the other is ? then the parent will be asigned A or B depending on the identity of the defined child. Using this pattern and working up the tree, eventually the children of the root will be assesed, thus determining the root label. The root label does not provide a difinitive donor identity but it is a way to gain a rough inferance.

The second summary statistic is the toplogical class. This is a method of assessing grouping patterns amongst nodes(\cite{doi:10.1073/pnas.1522930113}). The classes are assessed based on if the lineages are polyphyletic or monophyletic. Polyphyletic would mean that the lineages within the host appear on different branches of the tree, sperated by lineages from the other host. Monophyletic means that all of the lineages within that host are on the same branch of the tree, unseperated by lineages from other hosts. This means that, in this case where there are two hosts, the topological class is either PP where both hosts are polyphyletic, MP where one host is monophyletic and the other is polyphyletic, and MM where both are monophyletic. This allows for assessment of tree shape and distribution which gives insight to the patters of lineage transmission and thus helps determine direction.

The third summary statistic is the minimum number of transmissive lineages for each possible donor. This requires the consideration of the two scenerios where each host is the donor. From these scenarios, the minimum number of transmissive lineages is counted by observing branches which contain only the other host. The number of these diversions is the minimum number of transmissive lineages. This can help evaluate the likelyhood of each host being the donor as it would provide the amount of lineages that must have been transmitted if the host was the donor.

The fifth, sixth, and seventh statistics described were the average pairwise distances between tips with opposite labels, tips with A labels, and tips with B labels(A and B being the two hosts). This provides another measure for tree shape. This statistic can offer a comparison between distances which may lead to a coorilation to donor-recipient relations.

The fourth summary statistic is the number of nucleotide substitutions. This was not analysed in this paper as it would not be included in the given data. This paper is analysing summary statistics beginning at simulated trees and not from the genetic sequences. The eighth and nineth summary statistics are the mean and standard deviation in tree height. This is not directly useful to this research as the different simulation scenarios would always create different tree heights.


\begin{figure}
  \centering
  \includegraphics[width=0.4\textwidth, angle=0]{histories}
  \caption{Display of infection and sampling times across simulated sets of trees}
  \label{fig_mom0}%
\end{figure}

\section{Method}
\label{method}

The method used in this paper for the comparison and evaluation of the effectivness of each summary statistic was the use of a python program to apply each summary statistic to large sets of simulated trees(\cite{10.1371/journal.pcbi.1009741}). This would allow for a large data set of trees with known donors to be evaluated by each summary statistic and graphed in order to allow for easy comparison and evaluation of effectivness. This involved runing 23 sets of 50 trees through the program and plotting the collective data from each summary statistic to evaluate patterns. Each set had its own specific parameters such as the time gaps between when donor and recipient are infected and sampled(Figure 1). For example, set I had sample times of a year after infection. Set H had sample times that were a year apart. These all contribute to tree shape and structure as during that time, lineages form, die, and the structure changes. 

  
\begin{figure}
	\centering 
	\includegraphics[width=0.4\textwidth, angle=0]{sumstat1graph.pdf}	
	\caption{Display of the root labels for trees of different sets} 
	\label{fig_mom0}%
\end{figure}

\begin{figure}
	\centering 
	\includegraphics[width=0.4\textwidth, angle=0]{sumstat2graph.pdf}	
	\caption{Display of the topologics classes for trees of different sets} 
	\label{fig_mom0}%
\end{figure}

\begin{figure}
	\centering 
	\includegraphics[width=0.4\textwidth, angle=0]{sumstat3hist.pdf}	
	\caption{Histogram displaying the difference between lineages when D is donor vs when R is donor} 
	\label{fig_mom0}%
\end{figure}

\begin{figure}
	\centering 
	\includegraphics[width=0.4\textwidth, angle=0]{dd_rr_distance.pdf}	
	\caption{Histogram displaying the difference in average distance between two D nodes and two R nodes} 
	\label{fig_mom0}%
\end{figure}

\begin{figure}
	\centering 
	\includegraphics[width=0.4\textwidth, angle=0]{dr_dd_distance.pdf}	
	\caption{Histogram displaying the difference in average distance between a D and R node and two D nodes} 
	\label{fig_mom0}%
\end{figure}

\begin{figure}
	\centering 
	\includegraphics[width=0.4\textwidth, angle=0]{dr_rr_distance.pdf}	
	\caption{Histogram displaying the difference in average distance between a D and R node and two R nodes} 
	\label{fig_mom0}%
\end{figure}

\begin{figure}
  \centering
  \includegraphics[width=0.4\textwidth, angle=0]{correctsumstats}
  \caption{Bar graph which displays a comparison of how accurate each of the first 3 summary statistics are across all 23 sets of trees. Also shows a majority rules system counting all trees for which 2 of the 3 summary statistics are accurate.}
  \label{fig_mom0}%
\end{figure}

\section{Results and Discussion}
\label{results and discussion}

As seen in the data for summary statistic 1(Figure 2), overall the donor(D) appeared most often as the root label for almost all of the sets of trees. There are noteable exceptions in sets E, F, H, I, and J. In the data for summary statistic 2(Figure 3), the most frequent topological class for every data set is PP. However, a notable pattern is that when the topological class is not PP, it is most likely to be MP with the recipient being monophyletic and the donor being polyphyletic. While this is not constant across all trees, it does show coorilation. The data for summary statistic 3(Figure 4) shows an overall trend that more often than not, the true donor will have fewer transmissive lineages than when the recipient is the donor. However, this is only a general trend and is not consistant across the different sets, or even in some cases, within the sets themselves. The distance graphs(Figure 5, Figure 6, Figure 7) seem to indicate a similar distance difference across all three comparisons for each set. For example, set A has distance differences that are all almost completely 0. These similar distances are seen across almost all sets. If there is a pattern within these distance measurments it is not significant enough to disearn a donor, espcially when given only one tree.

Many of these results do seem to indicate a coorilation between the respective summary statistic and a donor-recipient relationship. However, because there are exceptions to each of the summary statistics, they cannot alone be used to determine infection direction. If one were to be given a single tree that had sample lineages from two hosts, it would be very difficult to tell if the tree was an exception to one of the summary statistic patterns. That is why the statistics must be used together as even if the tree was a statistical exception for one summary statistic, it may check the box on the others. The distance differences offer very little insight into direction of transmission. Because all of the distance differences are very similar and those that are not don't seem to show a difinitive pattern, they are relatively unhelpful in distinguishing transmission direction.

The way to use these statistics to find direction is to use them in combination. For example, the root label alone may leave a lot of doubt about the results. However, if you also test the topological class and the two happen to line up, that would create much more confidence in the results. By also testing the number of transmissive lineages one can become more and more certain that the recipient which has the most summary statistics supporting it being the donor is the true donor.

When a majority rules system was tested in which all trees that have 2 of the three first summary statistics returning an accurate donor estimation are counted(Figure 8). Intersetingly, this system did not prove to be more accurate. In fact, in many sets it was outpreformed by summary statistics 3. This leads to the conclusion that summary statistics 1 and 2 often tend to be inaccurate for trees with the same infection and sampling times. This means that when summary statistic 1 is inaccurate, summary statistic 2 usually is as well.

\section{Future Works}
\label{Future Works}
In terms of future developments concerning this subject, the creation of a new summary statistic which can create accurate results alone would be a logical future step. Additionally, a new way to count transmitted lineages other than the minnimum number could be used to create a summary statistic which does not fail under the same conditions as other summary statistics and thus offers, at the very least, a way of using it in a majority rules system. There also may be room for a more effective system of vote counting the summary statistics such as the use of a weight system which puts a heavier weight on certain summary statistics which may be more accurate for a larger amount of trees and a lighter weight on those that prove to be accurate less often.

\section{Acknowledgements}
\label{acknowledgements}
I would like to express sincere gratitude to my mentor Emma Elizabeth Goldberg of the Los Alamos National Laboratory for guiding me through this research process. I would also like to thank the Institute for Computing in Research for providing me with this research opportunity. 

%% The Appendices part is started with the command \appendix;
%% appendix sections are then done as normal sections

%% If you have bibdatabase file and want bibtex to generate the
%% bibitems, please use
%%
\bibliographystyle{elsarticle-harv} 
\bibliography{paperbibliography}

%% else use the following coding to input the bibitems directly in the
%% TeX file.

%%\begin{thebibliography}{00}

%% \bibitem[Author(year)]{label}
%% For example:

%% \bibitem[Aladro et al.(2015)]{Aladro15} Aladro, R., Martín, S., Riquelme, D., et al. 2015, \aas, 579, A101


%%\end{thebibliography}

\end{document}

\endinput
%%
%% End of file `elsarticle-template-harv.tex'.
